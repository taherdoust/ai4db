\section{Literature Review}
\label{sec:lit}

Research in Text-to-SQL has accelerated in recent years, largely driven by the introduction of cross-domain benchmarks such as Spider and BIRD \cite{bird}. BIRD in particular emphasizes execution accuracy and query efficiency, with results showing that frontier closed-source models like GPT-4 achieve over 80\% execution accuracy, while fine-tuned open-source models with up to 32B parameters have reached approximately 75\% \cite{survey_llm_t2sql_liu2025,survey_llm_t2sql_huang2025}. This performance gap highlights the dual trajectory of research in the field: the exploration of complex reasoning pipelines and agents on the one hand, and the development of fine-tuned specialized models on the other.

Agent-based and reasoning-augmented approaches attempt to decompose the problem of natural language to SQL into manageable subtasks. They rely on chain-of-thought prompting, tool-use, or multi-agent collaboration to analyze a question, map it to relevant schema components, draft a query, and iteratively repair errors through verification or execution feedback. Recent examples include retrieval-augmented and graph-enhanced systems that integrate schema reasoning more explicitly into the model’s workflow \cite{refsql,sqlfuse,grafix_t5,omnisql}. These strategies tend to improve compositional generalization but depend on sophisticated orchestration beyond the base language model.

Schema linking and retrieval-augmented generation (RAG) form a second major line of work. Here, the emphasis is on bridging natural language tokens with schema elements, often including table names, column names, and literal values. IRNet \cite{irnet} formalized schema linking as an explicit intermediate step, while later works extended it with relation-aware attention and value-based retrieval. More recently, systems such as ReFSQL and SQL-Fuse have integrated retrieval mechanisms with fine-tuning, thereby unifying schema awareness with model adaptation \cite{refsql,sqlfuse}. In spatial contexts, schema linking acquires additional complexity. Not only must the system identify geometry-bearing columns, but it must also map spatial expressions in the question, such as “within,” “near,” or “intersects,” to the corresponding PostGIS operators. Ontology-based or retrieval-augmented methods have been proposed to enhance this mapping, particularly by leveraging spatial hierarchies and domain ontologies \cite{spatial_rag,opensql}.

A third direction focuses on fine-tuning. Rather than relying solely on in-context learning, fine-tuned models specialize on large collections of NL–SQL pairs. This approach has proven particularly effective when combined with synthetic data generation and execution-guided filtering \cite{omnisql,refsql,sqlfuse,grafix_t5}. The BIRD leaderboard illustrates that while prompting with closed models can yield high absolute performance, open-source fine-tuned models can achieve competitive execution accuracy, thereby offering a reproducible and transparent alternative to proprietary systems. Despite this success in relational SQL, the application of fine-tuning to spatial SQL remains scarce.

In contrast to the vibrant progress in general Text-to-SQL, the Text-to-Spatial-SQL domain is still nascent. Early efforts focused on natural language interfaces to GIS or spatial databases, typically through rule-based methods that supported a narrow range of spatial operators. The most significant recent contribution is Wang et al.\ (2025) \cite{wang_spatial_ijgi}, who introduced a bilingual dataset (English and Chinese) covering four domains—Administrative Divisions, Education, Tourism, and Traffic—and evaluated GPT prompting over SpatiaLite. Their work represents the first attempt to formalize Text-to-Spatial-SQL evaluation, but it does not address fine-tuning of open-source models or schema-aware pipelines. Other works, such as OpenCity \cite{opencity} and CityLeo \cite{cityleo}, focus on urban simulation platforms and decision support, highlighting the demand for natural language interfaces to spatial databases but without providing standardized benchmarks or systematic evaluation.

Overall, the literature reveals a clear imbalance. Text-to-SQL is supported by multiple benchmarks, systematic surveys, and a wide variety of approaches that include agent-based reasoning, schema-linking and retrieval, and fine-tuning. Text-to-Spatial-SQL, on the other hand, lacks a comprehensive survey, a standardized benchmark comparable to BIRD or Spider, and large-scale exploration of fine-tuned open models. This gap motivates the present work, which contributes by reviewing the available literature and demonstrating a schema-aware prototype for CIM Wizard.